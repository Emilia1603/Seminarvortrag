

\documentclass{beamer} 
\usepackage{biblatex}
\usepackage{ mathrsfs }
\usepackage{mathtools}

\usepackage[ngerman]{babel}
\usepackage[utf8]{inputenc}


\addbibresource{Seminarvortrag.bib}
\makeatletter
\newcommand\listofframes{\@starttoc{lbf}}
\makeatother

\addtobeamertemplate{frametitle}{}{%
  \addcontentsline{lbf}{section}{\protect\makebox[2em][l]{%
    \protect\usebeamercolor[fg]{structure}\insertframenumber\hfill}%
  \insertframetitle\par}%
}

\setbeamercolor{block body}{bg=structure!10}
\setbeamercolor{block title}{bg=structure!20}

\title{Erweiterte Higgs Sektoren}
\author{Seminarvortrag - Emilia Welte}
\date{ \today}


\begin{document}
\begin{frame} 
\titlepage
\end{frame}


\begin{frame}
\frametitle{Gliederung}
\listofframes
\end{frame}

\begin{frame}
\frametitle{Zu klärende Fragen}
\begin{itemize}
\item Was sind erweiterte Higgs Sektoren ?
\item Warum braucht man erweiterte Higgs Sektoren ?
\end{itemize}
\end{frame}

\begin{frame}

\frametitle{SM Wiederholung - Eichsektor}
\begin{itemize}
\item Dynamik der Eichbosonen steckt in Form von Feldstärketensoren in der Lagrangedichte $\mathscr{L}_{\text{Eich}}$
\item Die Wechselwirkung der Eichbosonen mit Fermionen/Skalaren steht in der kovarianten Ableitung $\mathscr{D}_{\mu}$
\end{itemize}

\begin{block}{SM Eichstruktur}
\begin{equation}
\text{SU}(3)_{C}\times \text{SU}(2)_{L} \times \text{U}(1)_{Y} 
\end{equation}
\end{block}

\begin{itemize}
\item Massenterme treten in quadratischer Ordnung der Felder auf, dies ist nicht mehr Eichinvariant nach Einsetzen in $\mathscr{L}_{\text{Eich}}$ $\rightarrow$ ungebrochene Eichsymmetrie führt zu masselosen Eichbosonen
\end{itemize}

\end{frame}

\begin{frame}
\frametitle{SM Wiederholung - Fermionsektor}
\begin{itemize}
\item SM enthält 3 Generationen von händigen Fermionen Feldern mit jeweils unterschiedlichen Transformationseigenschaften 
\end{itemize}
\begin{block}{Allg. Fermionen Feld Lagrange}

\begin{equation}
\mathscr{L}_{\text{Fermion}}=\overline{\Psi} i \partial_{\mu} \gamma^{\mu} \Psi -\text{m}\overline{\Psi} \Psi 
\end{equation}
\end{block}
Dabei entspricht erster Term dem kinetischen Anteil und zweiter Term massen Anteil.  $\gamma^{\mu}$ entsprich den Dirac-Matrizen.
\end{frame}
\begin{frame}
\begin{itemize}
\item Unter Ausnutzung der Projektionsoperatoren für links- und rechtshändige Fermionen ($1=\text{P}_{R}^{2}+ \text{P}_{L}^{2}$) separiert der der kinetische Teil in händige Anteile und ist Eichinvariant 

\end{itemize}
\begin{block}{kinetischer Anteil}
\begin{equation}
\overline{\Psi} i \partial_{\mu} \gamma^{\mu} \Psi  \rightarrow \overline{\Psi}_{L} i \partial_{\mu} \gamma^{\mu} \Psi_{L} +\overline{\Psi}_{R} i \partial_{\mu} \gamma^{\mu} \Psi_{R} 
\end{equation}
\end{block}

\begin{itemize}
\item Unter Ausnutzung derselben Relation von den Projektionsoperatoren sieht man am Massenterm, dass hierbei die händigen Zustände mischen. Dieser ist also nicht Eichinvariant.
\end{itemize}

\begin{block}{Massen Term}
\begin{equation}
\text{m}\overline{\Psi} \Psi  \rightarrow \text{m}\overline{\Psi}_{R} \Psi_{L} +\text{m}\overline{\Psi}_{L} \Psi_{R} 
\end{equation}
\end{block}
\end{frame}
\begin{frame}
\frametitle{Zusammenfassung}
\begin{itemize}
\item Der Eichbosonen Masseterm ist nicht Eichinvariant und kann nicht ohne Weiteres in die Lagrangedichte  eingesetzt werden $\rightarrow$ Ohne Symmetriebrechung sind Eichbosonen also Masselos
\item Der Fermion Masseterm ist nicht Eichinvariant und kann wie der Bosonen Trem  nicht ohne Weiteres in die Lagrangedichte  eingesetzt werden $\rightarrow$ Ohne Symmetriebrechung sind Fermionen also Masselos
\end{itemize}
\end{frame}
\begin{frame}
\frametitle{SM Wiederholung - Higgs Mechanismus }
\begin{itemize}
\item Neuer Bestandteil der experimentell bestätigte Bosonenmassen Erklärt $\rightarrow$ Einführung eines skalaren $\text{SU}(2)_{L}$-Duplett Feldes was durch Higgs Mechanismus zu spontaner $\text{SU}(2)_{L}\times\text{U}(1)_{Y}$ Symmetriebrechung führt
\item Duplett hat Hypercharge $Y=\frac{1}{2}$ und ist ein Farb Singlett 
\end{itemize}

\begin{block}{Higgs Duplett}
\begin{equation}
\Phi=\left( \begin{array}{c} \Phi^{0}\\ \Phi^{+} \end{array}\right)= \frac{1}{\sqrt{2}}\left( \begin{array}{c} \Phi_{1}+i\Phi_{2}\\ \Phi_{3}+i\Phi_{4} \end{array}\right)
\end{equation}
\end{block}
Dabei entsprechen $\Phi_{j}$  normierten reellen Feldern wobei $j \in [1,4] $
\end{frame}

\begin{frame}
\frametitle{SM Wiederholung - Higgs Mechanismus }
\begin{itemize}
\item Unsere SM Lagrangedichte sieht dann wie folgt aus
\end{itemize}
\begin{block}{SM Higgs Lagrangedichte}
\begin{equation}
\mathscr{L}_{\Phi}=(\mathscr{D}_{\mu}\Phi)^{\dagger}(\mathscr{D}^{\mu}\Phi) - \text{V}(\Phi) + \mathscr{L}_{\text{Yukawa}}
\end{equation}
\end{block}
\begin{itemize}
\item Die Allgemeine Form eines Higgs Potentials könnte wie folgt aussehen 
\end{itemize}
\begin{equation}
 \text{V}(\Phi)=-\mu^{2} \Phi^{\dagger}\Phi +\lambda ( \Phi^{\dagger}\Phi)^{2}
\end{equation}
\end{frame}
\begin{frame}
\begin{itemize}
\item ist $-\mu^{2}<0$ und $\lambda>0$  das das Minimum des Potentials weg von $|\Phi|=0$ womit Vakuums/minimums Energie nicht mehr invariant unter $\text{SU}(2)_{L}\times\text{U}(1)_{Y}$ Symmetrie $\rightarrow$ Eich Symmetrie ist spontan gebrochen
\item sind beide größen positiv hat das Potential sein minimum bei  $|\Phi|=0$  und ist parabelförmig, elektroschwache Symmetrie ist dann ungebrochen
\item im Falle $\lambda<0$ ist das Potential ungebunden und es gibt keinen stabilen Vakuumszustand
\end{itemize}
\end{frame}

\begin{frame}
\frametitle{SM Wiederholung - Higgs Mechanismus }
Da wir wissen das der Vakuumszustand im Potential Minimum liegen muss, erhalten wir für den Vakuumserwartungswert $v=\sqrt{\frac{\mu^2}{\lambda}}$. Wir definieren unsere Felder so, dass die Erwartungswerte wie folgt aussehen $\langle \Phi_{3}\rangle=v$ und  $\langle \Phi_{1}\rangle=\langle \Phi_{2}\rangle=\langle \Phi_{2}\rangle=0$ . Zusätzlich addieren wir zu $\Phi_{3}$ ein Feld $h$ welches einen verschwindenden Erwartungswert hat. Umgeformt nach $\mu$ und eingesetzt in unser Potential erhalten wir:
\end{frame}

\begin{frame}
\frametitle{SM Wiederholung - Higgs Mechanismus }
Diese Form des Potentials wollen wir nun nutzen um Sie in eine Form der Massen und Wechselwirkung des Higgsteilchens umzuschreiben:
\begin{block}{Allgemeine Form der Massenmatrizen}
\begin{equation}
\left( \begin{array}{c} \Phi_{1}\\ \Phi_{2} \\ \Phi_{3} \\ \Phi_{4} \end{array}\right)^{\dagger} \left( \begin{array}{rrr} M_{11}&M_{12}&M_{13}&M_{14}\\M_{21}&M_{22}&M_{23}&M_{24} \\ M_{31}&M_{32}&M_{33}&M_{34}\\ M_{41}&M_{42}&M_{43}&M_{44}\end{array}\right) \left( \begin{array}{c} \Phi_{1}\\ \Phi_{2} \\ \Phi_{3} \\ \Phi_{4} \end{array}\right)
\end{equation}
\end{block}
\end{frame} 






\begin{frame}
\frametitle{Erweiterung des SM Higgs Sektors am Beispiel des 2HDM}
\end{frame}
\begin{frame} 
\frametitle{Quellen}
\printbibliography
\end{frame}

 \end{document}